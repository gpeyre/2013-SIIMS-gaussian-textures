%%%%%%%%%%%%%%%%%%%%%%%%%%%%%%%%%%%%%%%%%%%%%%%%%%%%%%%%%%%%%%%
%%%%%%%%%%%%%%%%%%%%%%%%%%%%%%%%%%%%%%%%%%%%%%%%%%%%%%%%%%%%%%%
%%%%%%%%%%%%%%%%%%%%%%%%%%%%%%%%%%%%%%%%%%%%%%%%%%%%%%%%%%%%%%%
%%%%%%%%%%%%%%%%%%%%%%%%%%%%%%%%%%%%%%%%%%%%%%%%%%%%%%%%%%%%%%%
\section{Barycenter Mixing of Several Gaussian Models}
\label{sec-barycenter}


This section extends the construction of Section~\ref{sec-geodesic} to the case of mixing an arbitrary number of Gaussian models. The computation is however more involved since no closed form is known for the OT barycenter.

%%%%%%%%%%%%%%%%%%%%%%%%%%%%%%%%%%%%%%%%%%%%%%%%%%%%%%%%%%%%
\subsection{Optimal Transport Barycenter of Gaussian Distributions}

Given a family of Gaussian distributions $( \distr_i )_{i \in I}$ and weights $\rho = (\rho_i)_{i \in I}$ with $\rho_i \geq 0, \sum_i \rho_i=1$, the barycenter according to some distance $d$ is defined as
\eql{\label{eq-dfn-barycenter}
	\distr_\rho = \uargmin{\distr} \sum_{i \in I} \rho_i d(\distr_i,\distr)^2.
}
Note that when $|I|=2$, one recovers the definition~\eqref{eq-dfn-path}. Note that $\rho \in (\RR^+)^{I}$ is now a vector of barycentric coordinates (which differs from Section~\ref{sec-geodesic} where $\rho$ was a single scalar). We now recall a result proved in \cite{Carlier_wasserstein_barycenter}.

\begin{proposition}\label{prop-bary-gaussian}
	If at least one of the $\Cov_{i}$ has full rank, then the OT barycenter $\distr_\rho$ of  $(\distr_i = \Nn(\Mean_i,\Cov_i))_{i \in I}$ exists, is unique, and is a Gaussian process $\Nn(\Mean_\rho,\Cov_\rho)$ where
	\eq{
		\Mean_\rho = \sum_{i \in I} \rho_i \Mean_i
	}
	and $\Cov_\rho$ is the unique solution of the following fixed point equation
	\eql{\label{eq-ot-barycenter}
		\Phi(\Cov_\rho) = \Cov_\rho
		\qwhereq
		\Phi(\Cov)=
		\sum_{i \in I} \rho_i \pa{\Cov^{1/2} \Cov_i \Cov^{1/2} }^{1/2}.
	}
\end{proposition}

An open problem is to determine under which hypothesis the barycenter is still unique, when all the covariances $\Cov_i$ are rank-deficient, which is the case when averaging SN models. In the case $|I|=2$, Proposition~\ref{prop-geod-gaussian} ensures that this is the case under a restriction on the kernel of the covariances. 

%  which is likely to be the case in a generic situation. We conjecture that a similar behavior still holds in the case $|I|>2$ and we did not encounter any case of non-uniqueness during our numerical simulations.

In Figure~\ref{fig-barycenter-gaussian}, we illustrate and compare the barycenters of three 2-D Gaussians (displayed at the vertexes of a triangle) (a),(d) for the Euclidean-distance, (b),(e) for the Fisher-Rao-distance, and (c),(f) for the OT-distance. 
Note that each edge of the triangle corresponds to a geodesic between two 2-D Gaussian distributions.
A covariance $\Cov$ is represented using its associated unit ball (an ellipse) $\enscond{x \in \RR^2}{ x^* \Cov x \leq 1 }$. The Euclidean barycenter does not maintain the rank-1 property, and ellipses obtained as barycenters are more ``round''. The Fisher-Rao barycenter is not defined for rank-1 covariances, and tends to become ill-posed as the covariance becomes rank deficient. In contrast, the OT barycenter maintains the rank-1 property along the edges of the triangle, and becomes full rank only near the center of the triangle.

\begin{figure}[ht!]
  \centering
  \subfloat[Euclidean barycenter]{
  \includegraphics[width=0.32 \linewidth]{gaussian-interp-eucl-1e-03.png}
  }
  \subfloat[Fisher-Rao-barycenter]{
  \includegraphics[width=0.32 \linewidth]{gaussian-interp-rao-1e-03.png}
  }
  \subfloat[OT-barycenter]{
  \includegraphics[width=0.32 \linewidth]{gaussian-interp-ot-1e-03.png}
  }\\

  \subfloat[Euclidean barycenter]{
  \includegraphics[width=0.32 \linewidth]{gaussian-interp-eucl-1e-01.png}
  }
  \subfloat[Fisher-Rao-barycenter]{
  \includegraphics[width=0.32 \linewidth]{gaussian-interp-rao-1e-01.png}
  }
  \subfloat[OT-barycenter]{
  \includegraphics[width=0.32 \linewidth]{gaussian-interp-ot-1e-01.png}
  }\\
  \caption{Comparison of different barycenters obtained from three 2-dimensional Gaussian distributions defined with the same mean vector but different covariance matrices, $\distr_i = \Nn(\Mean, \Cov_i), i=0,1,2$.
  (a, b, c) Barycenters of three Gaussian distributions with almost rank-1 covariance matrices. 
  (d, e, f) Barycenters of three full rank covariances.
  }
  \label{fig-barycenter-gaussian}
\end{figure}

%%%%%%%%%%%%%%%%%%%%%%%%%%%%%%%%%%%%%%%%%%%%%%%%%%%%%%%%%%%%
\subsection{Barycenter of SN Models}

%%
\paragraph{Fixed point equation}

The following proposition, which is a direct consequence of Proposition~\ref{prop-bary-gaussian}, describes the barycenter of SN models using a series of fixed point equation over the Fourier domain.

\begin{proposition}
	If $(\distr_i = \Nn(\Mean_i,\Cov_i))_{i \in I}$ are stationary Gaussian processes, a barycenter satisfies
	\eql{\label{eq:fp}
		\foralls \om \in U, \quad
		\hat \cov_\rho(\om) = \Phi_\om(\hat \cov_\rho(\om))
	}
	where
	\eq{
		\Phi_\om : c \in \CC^{d \times d}
		\mapsto
		\sum_{i \in I} \rho_i \pa{ c^{1/2} \hat \cov_i(\om) c^{1/2} }^{1/2} \in \CC^{d \times d}.
	}
\end{proposition}


We note that in general, $\distr_\rho$ is not spot noise because $\hat\cov_\rho(\om)$ is not necessarily rank one.

%%
\paragraph{Numerical scheme}

In general, put aside the case where the covariances $\hat \cov_i(\om)$ diagonalize in the same bases, or the case $|I|=2$ (see Theorem~\ref{thm-geodesic-spot-noise}), it is not possible to solve the fixed point equation~\eqref{eq:fp} in closed form.

Following \cite{knott-barycenter-gaussian}, we propose to approximate $\hat \cov_\rho(\om)$ by iterating the mapping $\Phi_\om$. Formally, for each $\om \in U$, we initialize $\hat \cov_\rho^{(0)}(\om)$ (for instance using $\hat \cov_i(\om)$ for some $i \in I$) and then iterate
\eql{\label{eq-numerical-scheme-barycenter}
	\foralls \ell \geq 0, \quad \hat \cov_\rho^{(\ell+1)}(\om) = \Phi_\om( \hat \cov_\rho^{(\ell)}(\om) ).
}
We observe numerically the convergence
\eq{
	\hat \cov_\rho^{(\ell)}(\om) \overset{\ell \rightarrow +\infty}{\longrightarrow}
	\hat \cov_\rho(\om)
}
where $\hat \cov_\rho(\om)$ is the Fourier transform of the kernel of a barycenter covariance $\Cov_\rho$.

Note that the mapping $\Phi_\om$ is not strictly contracting, but we conjecture that some iterate $\Phi_\om \circ \ldots \circ \Phi_\om$ of this mapping is contracting (as it is the case in dimension $d=1$), although we were not able to prove it in arbitrary dimension $d>1$.

The numerical computation of $\Phi_\om$ in the case $d=3$ requires the computation of the square root of $3 \times 3$ matrices, which is performed explicitly by computing the eigenvalues of the symmetric matrix as the root of a third order polynomial. Algorithm~\ref{alg-sn-barycenter} details the method.

\begin{algorithm}[ht!]
\caption{SN Barycenter Synthesis}
\label{alg-sn-barycenter}
% \begin{algorithmic}[1]
\Require exemplars $(\tilde f_i \in \RR^{\U \times d})_{i \in I}$, weight $(\rho_{i})_{i \in I} \in (\RR^+)^{I}$ with $\sum_i \rho_i=1$.
\Ensure sample $f \in \RR^{\tilde \U \times d}$ of the barycentric SN model.
% \Statex
\begin{enumerate}
	\algostep{Pre-processing} Compute $(f_i)_{i \in I}$ from $(\tilde f_i)_{i \in I}$
			using~\eqref{eq-preprocess}.
	\algostep{SN mixing} Compute $\Cov_\rho$ from $(\Cov_i)_{i \in I}$ using the iterations~\eqref{eq-numerical-scheme-barycenter}.
	\algostep{Texture synthesis} Apply steps 3, 4 and 5 of Algorithm~\ref{alg-gaussian-texture-modeling}
		with $\Cov_\rho$ \\ instead of $\Cov$.
\end{enumerate}
% \end{algorithmic}
\end{algorithm}


%%%%%%%%%%%%%%%%%%%%%%%%%%%%%%%%%%%%%%%%%%%%%%%%%%%%%%%%%%%%
\subsection{Barycenter of AR Models}

For any covariance $(\Cov_i)_{i \in I}$, let us denote
\eq{
	\Cov_\rho = \Bary_\rho( \Cov_i )_{i \in I}
}
where $\Cov_\rho$ is the covariance of the barycenter $\distr_\rho=\Nn(0,\Cov_\rho)$ of $(\distr_i=\Nn(0,\Cov_i))_{i \in I}$ with weight $\rho$, that minimizes~\eqref{eq-dfn-barycenter} and can be computed using the iterations of the mapping $\Phi$ defined in~\eqref{eq:fp}.

We compute an approximate OT barycenter of the AR models using the low-dimensional mapping exposed in Section~\ref{subsec-ar-geodesic}. This corresponds to computing the AR textons $(\Aa_\rho,\Bb_\rho)$ of the barycenter as
\eql{\label{eq-ar-mixing-multiple}
	\foralls \om \in \Ux, \; \foralls \rho \in (\RR^+)^I, \quad
	(\hat A_\rho(\om), \hat B_\rho(\om) ) = \Ga^{-1}\pa{
		\Bary_{\rho}( \Ga(\hat A_i(\om), \hat B_i(\om) )  )_{i \in I}
	}.
}
Algorithm~\ref{alg-ar-barycenter} details the whole process.

\begin{algorithm}[ht!]
\caption{AR Barycenter Synthesis}
\label{alg-ar-barycenter}
% \begin{algorithmic}[1]
\Require exemplars $(\tilde f_i \in \RR^{\U \times d})_{i \in I}$, weight $(\rho_{i})_{i \in I} \in (\RR^+)^{I}$ with $\sum_i \rho_i=1$.
\Ensure sample $f \in \RR^{\tilde \U \times d}$ of the barycentric AR model.
% \Statex
\begin{enumerate}
	\algostep{Pre-processing} Compute $(f_i)_{i \in I}$ from $(\tilde f_i)_{i \in I}$
			using~\eqref{eq-preprocess}.
	\algostep{AR texton learning} For $i \in I$, learn $(\Aa_i,\Bb_i)$ using~\eqref{eq-ar-texton-A} and~\eqref{eq-ar-texton-B}.
	\algostep{AR mixing} Compute the mixed parameter $(\Aa_\rho,\Bb_\rho)$ by applying~\eqref{eq-ar-mixing-multiple} \\ for each $\om \in \Ux$.
	\algostep{Texture synthesis} Apply steps 4 and 5 of Algorithm~\ref{alg-ar-modeling}
		with $(\Aa_\rho,\Bb_\rho)$ instead of $(\Aa,\Bb)$.
\end{enumerate}
% \end{algorithmic}
\end{algorithm}




